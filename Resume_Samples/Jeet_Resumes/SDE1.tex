
\documentclass[10.8pt, a4paper]{extarticle}
\usepackage{lipsum}
% \usepackage{fontspec}
% \setmainfont{Corbel}
\usepackage[T1]{fontenc}
\usepackage[utf8]{inputenc}
\usepackage{multicol}
\usepackage{bibentry}
\nobibliography*
\usepackage{romannum}
\usepackage{latexsym}
\usepackage[empty]{fullpage}
\usepackage{titlesec}
\usepackage{marvosym}
\usepackage[usenames,dvipsnames]{color}
\usepackage{verbatim}
\usepackage{enumitem}
\usepackage[pdftex, hidelinks]{hyperref}
\usepackage{fancyhdr}
\usepackage{fontawesome}
\setlist[itemize]{noitemsep, topsep=0.3pt}
\usepackage[charter]{mathdesign} % Bitstream Charter
% \usepackage{newpxtext,newpxmath} % Palatino

\pagestyle{fancy}
\fancyhf{} % clear all header and footer fields
\fancyfoot{}
\renewcommand{\headrulewidth}{0pt}
\renewcommand{\footrulewidth}{0pt}

% Adjust margins
\addtolength{\oddsidemargin}{-0.50in}
\addtolength{\evensidemargin}{-0.250in}
\addtolength{\textwidth}{1in}
\addtolength{\topmargin}{-.5in}
\addtolength{\textheight}{1.0in}
\addtolength{\parskip}{.5mm}
\urlstyle{same}

\raggedbottom
\raggedright
\setlength{\tabcolsep}{0in}
\usepackage[margin=0.3in]{geometry}
% Sections formatting
\titleformat{\section}{
  \vspace{-6pt}\scshape\raggedright\large
}{}{0em}{}[\color{black}\titlerule \vspace{-5pt}]

%-------------------------
% Custom commands
\newcommand{\resumeItem}[2]{
  \item\small{
    \textbf{#1}{: #2 \vspace{-2pt}}
  }
}

\newcommand{\resumeItemNoBullet}[2]{
  \item[]\small{
    \hspace{-9pt}\textbf{#1}{: #2 \vspace{-6pt}}
  }
}

\newcommand{\resumeSubheading}[4]{
  \vspace{-1pt}\item[]
  \begin{tabular*}{0.98\textwidth}{l@{\extracolsep{\fill}}r}
      \hspace{-10pt}\textbf{#1} & #2 \\
      \hspace{-10pt}\textit{\small#3} & \textit{\small #4} \\
    \end{tabular*}\vspace{-5pt}
}

\newcommand{\resumeSubItem}[2]{\resumeItem{#1}{#2}\vspace{-4pt}}

\renewcommand{\labelitemii}{$\circ$}

\newcommand{\resumeSubHeadingListStart}{\begin{itemize}[leftmargin=*]}
\newcommand{\resumeSubHeadingListEnd}{\end{itemize}}
\newcommand{\resumeItemListStart}{\begin{itemize}}
\newcommand{\resumeItemListEnd}{\end{itemize}\vspace{-5pt}}

% custom commands
\newcommand{\shorterSection}[1]{\vspace{-10pt}\section{#1}}

%-------------------------------------------1720 S Michigan Ave, Chicago, IL
%%%%%%  CV STARTS HERE  %%%%%%%%%%%%%%%%%%%%%%%%%%%%

\renewcommand{\baselinestretch}{1.0} 
\begin{document}

% \fontfamily{qhv}\selectfont
% \setlength{\parskip}{0em}

% \setmainfont{CORBEL.ttf}
\fontsize{9.8pt}{11.3pt}\selectfont
% \renewcommand{\baselinestretch}{1.0}
%----------HEADING-----------------
\vspace{0pt}
\begin{flushleft}
\noindent {\huge\textbf{Jeet Sarangi}}
\vspace{-5pt}
\end{flushleft}
Department of Computer Science $\&$ Engineering  \hfill \href{mailto:jeets21@iitk.ac.in}{\faEnvelope{ jeets21@iitk.ac.in}} / \faPhone{ +91-8249690490}
\\Indian Institute of Technology, Kanpur \hfill\href{https://github.com/jeetsarangi}{ \faGithub{ jeetsarangi}} / \href{https://www.linkedin.com/in/jeetsarangi/}{ \faLinkedinSquare{ jeetsarangi}}
\vspace{5pt}
\vskip -2mm  
\vskip -2mm
\rule{\textwidth}{1pt}

%-----------EDUCATION-----------------
\shorterSection{Education}
\begin{center}
\begin{tabular}{|p{2.5cm}|p{7.0cm}|p{7.5cm}|p{2.0cm}|}
\hline
\centering{\textbf{Year}} & \centering{\textbf{Degree/Certificate}} & \centering{\textbf{Institute}} & \ \ \ \ \ \ \textbf{CPI/$\%$}\\
\hline
\centering{2021-Present} & \centering{M.Tech/Computer Science \& Engg.} & \centering{Indian Institute of Technology, Kanpur} & \ \ \ \ \ 9.33/10\\
\hline
\centering{2016-2020} & \centering{B.Tech/Computer Science \& Engg.} & \centering{Biju Patnaik University of Technology,Rourkela} & \ \ \ \ \  8.54/10\\
% \hline
% \centering{2014-2016} & \centering{CBSE(\Romannum{12})} & \centering{Deepika English Medium School,Rourkela} & \ \ \ \ \ 51.2$\%$\\
% \hline
% \centering{2013-2014} & \centering{CBSE(\Romannum{10})} & \centering{Deepika English Medium School,Rourkela} & \ \ \ \ \ 8.0/10\\
\hline
\end{tabular}
\end{center}
\vspace{-1mm}



%-----------Internship-----------------
\shorterSection{Research Experience}
\vspace{-2pt}
\begin{itemize}
\item \textbf{Intelligent Program Indexing}(M.Tech Thesis)
\hfill\hfill(\textit{Mar'22 - Present})

Guides: Prof. Purushottam Kar \& Prof. Amey Karkare
\begin{itemize}
          \item[$\circ$] Building an AI-assisted tool for \textbf{Indexing C programming problems} based on conceptual tags\\[-0.6cm]
          
          \item[$\circ$] Designing a model for \textbf{Classification of hot areas} in problems and \textbf{adopting a context-sensitive approach} for extracting relevant portions\\[-0.6cm]
          \item[$\circ$] Exploring \textbf {Deep learning based Techniques} for \textbf {clustering related portions} and \textbf{multi-label classification}\\[-0.6cm]
          \item[$\circ$] Integrating the Indexing tool to Prutor an Intelligent E-tutoring platform for teaching programming courses\\[-0.6cm]
          \item[$\circ$] \textbf{Research Areas :} Multi-label classification, Feature Engineering
          
    \end{itemize}
\end{itemize}
\medskip



%-----------PROJECT-----------------
\shorterSection{Academic Projects}
\vspace{-2pt}
\begin{itemize}




 \item \href{https://github.com/jeetsarangi/Automated-Program-Analysis-Verification-and-Testing} {\textbf{Program Analysis, Verification and Testing on Turtle Framework}} (CS639) Guide: Prof.  Subhajit Roy  \hfill(Aug'21 - Nov'21)
	\\[-0.6cm]
	\begin{itemize}
	    \item[$\circ$] Implemented a tool that parse the \textbf{Abstract Syntax Tree} to report all Assignment Statements and conditions\\[-0.6cm]
	    
	   % Implemented a tool that parse AST to report all Assignment Statements,loop conditions and branch conditions.
	    \item[$\circ$] Implemented a tool to generate \textbf{Control Flow Graph} from given IR, used \textbf{Data Flow Analysis} to optimize given IR\\[-0.6cm]
	    
	   % \item[$\circ$] Reduced movements of turtle by optimizing the Intermediate Representation using \textbf{Data Flow Analysis}.\\[-0.6cm]
	    
	    \item[$\circ$] Synthesized unknown constants in a program using \textbf{Symbolic Execution} to make programs equivalent \\[-0.6cm]
	    
	    \item[$\circ$] Implemented a tool to perform verification of a property using \textbf{Abstract Interpretation}\\[-0.6cm]
	    
	    \item[$\circ$] Implemented \textbf{Coverage Guided Fuzzing} with custom mutation operator to find coverage
	\end{itemize}


% \item \textbf{Music Genre Classification}(CS771), Guide : Prof. Piyush Rai \hfill (\textit{Sept '20-Dec '20})
% \begin{itemize}
% \item[$\circ$] Used a subset of the GTZAN dataset which consisted of 10 genres each having 100 audio tracks of 30 secs.
% \item[$\circ$] Used Chroma Short-time Fourier transform, Spectral Centroid, and Spectral Contrast, as features.
% \item[$\circ$] Implemented a hierarchical LSTM architecture to perform the multi-class classification.
% \end{itemize}

\item\href{https://github.com/jeetsarangi/Wikipedia-Article-Retrieval-System} {\textbf{Linux PCI Device Driver}} (CS614) Guide: Prof.  Debadatta Mishra   \hfill(Feb'22 - Apr'22)
	\\[-0.6cm]
	\begin{itemize}
	
	\item [$\circ$] Implemented PCI Device Driver with support for \textbf{Memory-Mapped I/O(MMIO)}, \textbf{Direct Memory Access(DMA)} with and without interrupt for data transfer.\\[-0.6cm]
		
	\item [$\circ$] Implemented the \textbf{concurrency control} mechanism to handle \textbf{multiprocess} and \textbf{multithreaded} requests
	
    
    \end{itemize}

\item\href{https://github.com/jeetsarangi/Wikipedia-Article-Retrieval-System} {\textbf{Wikipedia Search Engine for English and Hindi}} (CS657) Guide: Prof.  Arnab Bhattacharya   \hfill(Feb'22 - Apr'22)
	\\[-0.6cm]
	\begin{itemize}
	
	\item [$\circ$] Pre-processed and parsed the Wikipedia XML-Dump using \textbf{regular expression matching and SAX parser}\\[-0.6cm]
	
% 	\item [$\circ$] Parsed Wikipedia XML dump for Hindi and English,prepared index pages with keywords using SAX parser.\\[-0.6cm]
	
	\item [$\circ$] Implemented \textbf{TF-iDF based Indexer} from scratch for \textbf{Ranking} based on similarity scores\\[-0.6cm]
	
	\item [$\circ$] Implemented query search efficiently by applying \textbf{indexing} on pre-processed Posting list
	 
    
    \end{itemize}

 \item\href{https://github.com/jeetsarangi/Machine-Learning-CS771-}{\textbf{Implementing ML Models}} (CS771) Guide: Prof. Nisheeth Srivastava \hfill(Aug'21 - Nov'21)
    \\[-0.6cm]
	\begin{itemize}
	   %   \item [$\circ$]Implemented K-NN based Regressor from scratch and trained it on UCI Automobile Data Set.\\[-0.6cm]
	   
	   %\item [$\cric$] Implemented K-NN Regression and Decision Tree Classification model  from scratch and trained on UCI Automobile and Adult Datasets.\\[-0.6cm]
	      \item [$\circ$]Implemented \textbf{K-NN Regression} and \textbf{Decision Tree Classification} model  from scratch and trained on UCI's Datasets\\[-0.6cm]
	      \item [$\circ$] Implemented \textbf {Perceptron Algorithm} and \textbf {variations of Gradient Descent} algorithms from scratch\\[-0.6cm]
	      \item [$\circ$]Using Gaussian Kernel implemented K-means and implemented \textbf {Expectation-Maximization} on Synthetic data\\[-0.6cm]
	      
	   %   \item [$\circ$]Implemented Expectation-Maximization
    %     for Gaussian Mixture Model on Synthetic data.\\[-0.6cm]
         
        %  \item [$\circ$] Implemented Gradient Descent ,Stochastic GD and Mini-batch GD form scratch.\\[-0.6cm]
         
         \item [$\circ$] Implemented \textbf {MCMC sampling} to approximate Bayesian Posterior\\[-0.6cm]
	\end{itemize}

\item \href{https://github.com/jeetsarangi/Network-Attack-Detection-and-Prevention-using-TCPdump-Grafana-and-Prometheus}{\textbf{Network Attack Detection and Prevention}} (CS659) Guide: Prof. Sandeep Kumar Shukla \hfill(Feb'22 - Apr'22)
	\\[-0.6cm]
	\begin{itemize}
	    \item[$\circ$] Developed a centralized NIDS to detect attacks like \textbf{DDoS, TCP SYN flooding, ARP spoofing, Smurf DOS} and \textbf{Nmap Scan}\\[-0.6cm]
	   % \item[$\circ$] TCPDump and Prometheus were used for \textbf {Network Monitoring} and Data collection\\[-0.6cm]
	    \item[$\circ$] Performed \textbf{Network Monitoring, Data collection} using TCPDump and Prometheus\\[-0.6cm]
	   % \item[$\circ$] Used Grafana for Visualisations\\[-0.6cm]
	    \item[$\circ$] Utilized Grafana to Create Visualizations for the Network's Health\\[-0.6cm]
	   % \item[$\circ$] Grafana was used for Visualisation and generating attack triggers
	\end{itemize}

\item \href{https://github.com/jeetsarangi/Modern-Cryptology}{\textbf{Escaping the Caves}} (CS641) Guide: Prof. Manindra Agrawal \hfill(Feb'22 - Apr'22)
    \\[-0.6cm]
	\begin{itemize}
	      \item [$\circ$] \textbf {Analyzed and Decoded} various cryptosystems namely, \textbf {Substitution cipher, PlayFair cipher, EAEAE,  DES, RSA, Hashing}\\[-0.6cm]
	      
	      \item [$\circ$] Exploited cryptosystems using techniques like \textbf {frequency analysis, differential cryptanalysis, coppersmith algorithm}\\[-0.6cm]
	      
	   %   \item [$\circ$] Performed cryptanalysis on several encryption algorithms like Substitution cipher, PlayFair cipher, DES, EAEAE, RSA,Hashing.
	\end{itemize}



	% \item \href{https://github.com/jeetsarangi/Accidents-and-suicides-in-India-Analysis}{\textbf{Analysis on Accidents and Suicides in India}}, (CS639) Guide: Prof. Arnab Bhattacharya \hfill(Aug'21 - Nov'21)
	% \\[-0.6cm]
	% \begin{itemize}
	%       \item [$\circ$] Pre-processed Data published by NCRB,converted PDF tables to CSV ,merged tables from various sources to CSV.\\[-0.6cm]
	%       \item [$\circ$] Performed Exploratory analysis by plotting several visualisations to generate insights on year wise trends across Indian states. \\[-0.6cm]
	      
	% \end{itemize}


\medskip

\end{itemize}
\vspace{-2mm}


\shorterSection{Other Projects}
\vspace{-2pt}
\begin{itemize}
     \item\href{https://github.com/jeetsarangi/spade}{\textbf{Spade(Spare Parts Demand Forecasting System)(Self)}}\hfill
    \\[-0.6cm]
    \begin{itemize}
	      \item [$\circ$]Built a \textbf {Web-Based Car Inventory Management System}, implemented utilities like authentication, account management, order management and cart management using jsp and servlets\\[-0.6cm]
	   %   \item [$\circ$]Implemented utilities like authentication,account management,order management and cart management.\\[-0.6cm]
	      \item [$\circ$] Designed and Implemented \textbf{Relational Tables} using Oracle Database
	\end{itemize}
	\end{itemize}
\medskip

%------------Achievements--------------

\shorterSection{Scholastic Achievements and Extra-Curricular}
\begin{itemize}
  \item Received the \textbf{Academic Excellence Award} for exceptional academic performance in 2021-22 academic session\\[-0.6cm]
  \item Secured \textbf{All India Rank 378} in GATE CS 2021, about 1 lakh candidates appeared for the examination\\[-0.6cm]
  \item \textbf{Honours} in Undergraduate under BPUT for extra credits\\[-0.6cm]
  \item Completed Neural Networks and Deep Learning course on coursera
\end{itemize}
\medskip


%-----------Position of Responsibility-----------------
\shorterSection{Positions of Responsibility}
\begin{itemize}
\item \textbf{Teaching Assistant :} Intro To Machine Learning (CS771) \hfill\hfill(\textit{Aug '22-Present})
% \item \textbf{Teaching Assistant :} Introduction to Python Programming \hfill\hfill(\textit{Jul '21-Aug '21})
% \item \textbf{Teaching Assistant :} Fundamental of Computing \hfill\hfill(\textit{Nov '20-Mar '21}) (\textit{Mar '21-Jul '21})
\vspace{2mm}
\end{itemize}
\vspace{-2mm}
% -----------Courses and Skills-----------------
% \shorterSection{Relevant Courses}
% \begin{tabular}{p{8.4cm}p{6.5cm}p{5.5cm}}
% \ \ \ \ \ \ \ \ \ \ Introduction to Machine Learning  &  Advanced Computer Architecture  &  Operating Systems
% \\
% \ \ \ \ \ \ \ \ \ \ Statistical Natural Language Processing   & Object Oriented Programming  &  Data Structures \& Algorithms
% \\
% \ \ \ \ \ \ \ \ \ \   Computational Genomics   & Data Mining  & Computer Networks 
% \\
% \ \ \ \ \ \ \ \ \ \   Database Management System   & Design and Analysis of Algorithms  & Mathematics 
% \end{tabular}






% % \shorterSection{Relevant Courses}
% % \begin{itemize}
% % \item \textbf{Postgraduate :} Introduction to Machine Learning, Statistical Natural Language Processing, Adv. Computer Architecture.
% % \item \textbf{Undergraduate :} Data Structures and Algorithms, Operating Systems, Databases, Computer Networks, Mathematics.
% % \end{itemize}
% %-----------TECHNICAL SKILLS-----------------
% \shorterSection{Technical Skills}
% \begin{itemize}
% \item \textbf{Languages :} C, C++, Python.
% \item \textbf{Languages :} C, C++, Python, Java.
% \item \textbf{ML Libraries:} Scikit-learn, Tensorflow, PyTorch.
% \item \textbf{Familiar :} Numpy, Pandas.
% \item \textbf{Utilities :} Git, \LaTeX, SQL.

% \vspace{2mm}
% \end{itemize}

\shorterSection{Relevant Courses and Technical Skills}
\begin{itemize}
\item \textbf{Mtech Courses :} Introduction to ML, Program Analysis Verification and Testing, Information Retrieval, Data Mining
\item \textbf{Btech Courses :} Data Structures \& Algorithms, Operating Systems, Computer Networks, Database Management System
\item \textbf{Languages/ML Libraries/Utilities :} C, C++, Python, Javascript, Scikit-learn,Pandas, Numpy, PyTorch, SQL, Git
\end{itemize}


% \vspace{-2mm}



\end{document}

% %-----------Addtional Experience & Achievements-----------------
% \shorterSection{Additional Experience \& Achievements}
%   \resumeSubHeadingListStart
%   \small
%     \item{Presented poster on \textit{Tiramisu DenseNet Architecture for Precise Segmentation} for Intel AI at \textbf{CVPR 2018}}
%     \vspace{-5pt}
%     \item{Selected as an \textbf{Intel AI Student Ambassador} (only 150 students) to research, publish, and share work on machine learning and deep learning}
%     \vspace{-5pt}
%     \item{Won \textit{Best Microsoft Hack} out of 220 teams at \textbf{HackHarvard 2017}}
%     \vspace{-5pt}
%     \item{Placed 16/50 at Google Games: Campus Edition 2017 at UIC}
%     \vspace{-5pt}
%     \item{Won \textit{Best Technical Innovation} award (out of 800 students) at \textbf{Amity University Convocation 2017}}
%     \vspace{-5pt}
%     \item{Elected as a \textit{Vice-Chair} for \textbf{ACM Amity Student Chapter} out of 800 students at Amity University based on high-achieving and technically strong undergraduate students}
%   \resumeSubHeadingListEnd
% %-------------------------------------------
% \end{document}